% A ConTeXt document [master document: literateProgs.tex]

\startchapter[title=The capture and manipulation of code]

All of our literate programming code capture environments are enhanced 
versions of the \ConTeXt\ \type{typing} environment. When creating 
\quote{standard} \ConTeXt\ environments the two commands, 
\type{\defineXXX} and \type{\setupXXX}, are critical to their overall use. 

\startTestSuite[defineLitProgs]

We begin by defining the \type{\defineLitProgs} command. This command is a 
simple enhancement of the \type{\definetyping} command, so we, 
essentially, define the \type{\defineLitProgs} command to call the 
\type{\definetyping} command.

However, the \type{\definetyping} command takes a variable number of up to 
three of arguments. This means that our definition of 
\type{\defineLitProgs} must also be able to deal with the same number of 
variable arguments. We use \ConTeXt's \type{\dotripleempty} command to 
provide the simplest and most explicit way of dealing with up to three 
variable arguments. A side-effect of our explicit pattern is that we 
provide the user with more appropriate error messages. 

The \type{\dotripleempty} command scans the token stream expecting a 
command plus up to three \quote{[} \quote{]} delimited arguments. It then 
sets one of the \type{\ifXXXargument} tests to true, depending upon how 
many arguments it has found. The \quote{real} work gets done in either the 
\type{\defineLitProgsSingle}, \type{\defineLitProgsDouble} or 
\type{\defineLitProgsTriple} commands, which explicitly expect either one, 
two or three arguments respectively. The \type{\defineLitProgsZero} 
command provides an appropriate error message, if the user has provided no 
arguments. The \type{\defineLitProgsDirect} command simply manages the 
choice of the appropriate \type{\defineLitProgsXXX} command. 

Since we are enhancing the \type{typing} environment using lua code, the 
\type{\defineLitProgsSingle}, \type{\defineLitProgsDouble} and 
\type{\defineLitProgsTriple} commands make use of the 
\type{\fixStartLitProgs} and \type{\fixStopLitProgs} commands to layer in 
our enhancements. These two commands will be defined below. Similarly, we 
need to define a setup command corresponding to our new code type. We do 
this using the \type{defineLitProgsSetup} command, also to be defined 
below. Finally for each code type we would like a corresponding 
\type{createXXXFile} macro to write the accumulated text for a given code 
type out into a file. 

\startMkIVCode
\unexpanded\def\defineLitProgs{%
  \dotripleempty\defineLitProgsDirect%
}

\unexpanded\def\defineLitProgsDirect{%
  \ifthirdargument%
    \noexpand\defineLitProgsTriple%
  \else\ifsecondargument%
    \noexpand\defineLitProgsDouble%
  \else\iffirstargument%
    \noexpand\defineLitProgsSingle%
  \else%
    \noexpand\defineLitProgsZero%
  \fi\fi\fi%
}

\unexpanded\def\defineLitProgsTriple[#1][#2][#3]{
  \definetyping[#1][#2][#3]
  \fixStartLitProgs{#1}
  \fixStopLitProgs{#1}
  \defineLitProgsCreateFile{#1}
  %\defineLitProgsSetup{#1}
}

\unexpanded\def\defineLitProgsDouble[#1][#2][#3]{
  \definetyping[#1][#2]
  \fixStartLitProgs{#1}
  \fixStopLitProgs{#1}
  \defineLitProgsCreateFile{#1}
  %\defineLitProgsSetup{#1}
}

\unexpanded\def\defineLitProgsSingle[#1][#2][#3]{
  \definetyping[#1]
  \fixStartLitProgs{#1}
  \fixStopLitProgs{#1}
  \defineLitProgsCreateFile{#1}
  %\defineLitProgsSetup{#1}
}

\unexpanded\def\defineLitProgsZero[#1][#2][#3]{
  \errmessage{
    \string\\defineLitProgs
    requires at least one argument,
    you have provided none.
  }
}
\stopMkIVCode

\startTestCase[should call all associated macros]

We assert that the \type{\defineLitProgs} macro actually defines all of 
the required macros. 

\startConTest
\startMocking
  \defContextMock{defineLitProgsDirect}
  \defineLitProgs[TestCode]
  \assertMockExpanded{defineLitProgsDirect}{}
\stopMocking
\stopConTest

\startConTest
\startMocking
\startTracingMockExpansions
  \defContextMock{defineLitProgsTriple}
  \meaning\defineLitProgsTriple
  \defContextMock{defineLitProgsDouble}
  \meaning\defineLitProgsDouble
  \defContextMock{defineLitProgsSingle}
  \meaning\defineLitProgsSingle
  \defContextMock{defineLitProgsZero}
  \meaning\defineLitProgsZero
  \firstargumentfalse
  \secondargumentfalse
  \thirdargumentfalse
  \assertNoFirstArgument{first}
  \assertNoSecondArgument{first}
  \assertNoThirdArgument{first}
  \defineLitProgs[AFirstArg]
  \assertFirstArgument{second}
  \assertNoSecondArgument{second}
  \assertNoThirdArgument{second}
  \assertMockExpanded{defineLitProgsSingle}{}
\stopTracingMockExpansions
\stopMocking
\stopConTest

\startConTest
\startMocking
  \defTexMockOneArg{fixStartLitProgs}
  \defTexMockOneArg{fixStopLitProgs}
  \defTexMockOneArg{defineLitProgsCreateFile}
  \defTexMockOneArg{defineLitProgsSetup}

  \defineLitProgs[TestCode]
  \assertDefined{startTestCode}{}
  \assertDefined{stopTestCode}{}
  \assertMockExpanded{fixStartLitProgs}{}
  \assertMockExpanded{fixStopLitProgs}{}
  \assertMockExpanded{defineLitProgsCreateFile}{}
  %\assertMockExpanded{defineLitProgsSetup}{}
\stopMocking
\stopConTest

\stopTestCase

\startTestCase[should define associated marcros]

\startConTest
%  \defineLitProgs[TestCode]
  \assertDefined{startTestCode}{}
  \assertDefined{stopTestCode}{}
\stopConTest

\stopTestCase

\stopTestSuite

Having provided the \ConTeXt\ code required to deal with variable 
arguments, we now need to provide the code required to \quote{fix} the 
\type{\startXXX} and \type{\stopXXX} commands automatically generated by 
the \type{\defintyping} command. We do this with the 
\type{\fixStartLitProgs} and \type{\fixStopLitProgs} commands. 

Both of these commands essentially \quote{wrap} the original 
\type{\startXXX} and \type{\stopXXX} commands with calls to an appropriate 
lua function (defined below). To do this our \quote{fix} commands 
\type{\let} an \type{\oldXXX} command to \emph{be} the original command, 
and then redefine the original command to make the call into the lua code 
and then call the new \type{\oldXXX} command. Order here is critical as 
the \type{typing} environment is scanning for the exact token 
\type{\stopXXX}. Since all of these new commands are generated from the 
user's supplied code name, we must make use of the 
\type{\csname\endcsname} code sequence to dynamically build the command 
names. 

A final complication is that our \type{\startXXX} command can take an 
optional argument which specifies a particular stream of code. If this 
optional argument is not provided, the 'default' stream will be used. We 
use an explicit variable argument pattern similar to that used to define 
\type{\defineLitProgs} above. 

\startMkIVCode
\def\fixStartLitProgs#1{%
  \letvalue{oldStart#1}=\getvalue{start#1}%
  \setuvalue{start#1}{%
    \dosingleempty\getvalue{start#1Direct}%
  }
  \setvalue{start#1Direct}{%
    \iffirstargument%
      \getvalue{start#1Single}%
    \else
      \getvalue{start#1Zero}%
    \fi
  }
  \setvalue{start#1Zero}{%
    \getvalue{start#1Single}[default]%
  }
  \setvalue{start#1Single}[#1]{%
    \directlua{thirddata.literateProgs.setCodeStream('#1')}%
    \getvalue{oldStart#1}%
  }
}
\stopMkIVCode

Having defined \type{\fixStartLitProgs}, we have a much simpler task to do 
more or less the same technique to fix the \type{\stopXXX} command. 
However, in this case, things are simpler since the \type{\stopXXX} 
command takes no arguments. 

\startMkIVCode
\def\fixStopLitProgs#1{%
  \letvalue{oldStop#1}=\getvalue{stop#1}%
  \setvalue{stop#1}{%
    \getvalue{oldStop#1}%
    \directlua{
      thirddata.literateProgs.addCode(
        '#1',
        '_typing_'
      )
    }
  }
}
\stopMkIVCode

% Do we need the following?

% Having defined a new code type in the previous section, we now have the 
% very much simpler task of providing a \type{\setupXXX} command. At the 
% moment we do not provide any extra options other than those provided by 
% the \type{\setuptyping} command. This means that all we have to do is to 
% define the \type{\setupXXX} to be a simple call to the 
% \type{\setuptyping} command. In all cases the \type{\setupXXX} will have 
% a single manditory argument which provides the collection 
% \type{\setuptyping} options to be used by this code type. 

%\startMkIVCode
%\def\defineLitProgsSetup#1{%
%  \def\csname setup#1\endcsname{%
%    \setuptyping[#1]%
%  }
%}
%\stopMkIVCode

Finally, we need a code file creation command for each code type, 
\type{\createXXXFile}. This command takes two explicit arguments, the 
first argument is the name of the code stream, this is the same string as 
used in any of the corresponding \type{\startXXX} commands. The second 
argument is the name of the code file to be created. 

\startMkIVCode
\def\callCreateCodeFile#1#2#3{%
  \directlua{
    thirddata.literateProgs.createCodeFile(
      '#1',
      '#2',
      '#3'
    )
  }
}
\def\defineLitProgsCreateFile#1{%
  \setevalue{create#1File}{%
    \noexpand\callCreateCodeFile #1
  }
}
\stopMkIVCode

We now turn to the lua code used in the \type{\directlua} commands above. 
As defined in the preamble, \type{litProgs} and \type{code} are variables 
local to the \type{t-literateProgs.lua} file. They provide access to the 
collection of lua functions and code fragments respectively. 

\startLuaCode
function litProgs.setCodeStream(aCodeStream)
  code.curCodeStream = aCodeStream
end

function litProgs.addCode(aCodeType, bufferName)
  local bufferContents  =
    buffers.getcontent(bufferName):gsub("\13", "\n")
  code[aCodeType]       = code[aCodeType] or { }
  local codeType        = code[aCodeType]
  local aCodeStream     = code.curCodeStream or 'default'
  codeType[aCodeStream] = codeType[aCodeStream] or { }
  local codeStream      = codeType[aCodeStream]
  tInsert(codeStream, bufferContents)
end

function litProgs.createCodeFile(aCodeType,
                                 aCodeStream,
                                 aFilePath)
  -- here be dragons! -- how do we pass in cType and cSubType
  renderFile(aFilePath, litProgs.templates.lua.file)
  -- here be dragons!
end
\stopLuaCode

\stopchapter