% A ConTeXt document [master document: literateProgs.tex]

\startchapter[title=A Lua Rendering Engine]

We follow Terance Parr's advice, \cite{parr2004templateMVC}, to ensure 
there is a clear separation between Model and View when rendering 
templates. In his view, the rendering process is a form of 
\quote{unparsing} or, alternatively, language generation. Using Parr's 
\quote{push} semantics, a template is defined to be a literal string with 
embedded pairs of \quote{\{\{}, \quote{\}\}} delimited \quote{holes} or 
expansion actions, where each expansion action must not have side-effects 
on the data model. 

Our template engine will support two semantic actions: simple attribute 
reference and mapping a named template against a list of attribute values. 
To do this we will have three forms of expansion actions: 

\startitemize
\item {\bf reference}: \type{{{= <attribute-list> }}} 

\item {\bf application}: \type{{{! <attribute-list> }}}

\item {\bf mapping}: \type{{{| <attribute-list> }}}

\stopitemize

Note that attribute application is simply a convenient form of attribute 
mapping. Note as well that while attribute mapping is a form of \type{for} 
\type{in} looping, it, and attribute application, can be used to replace 
any need for explicit \type{if}, \type{elseif}, \type{else}, \type{end} 
structures. In all cases, if an attribute value is a Lua \type{nil}, there 
will be no template expansion. 

In each case, the \type{<attribute-list>} is a comma separated list of 
attribute names or indirect references. An indirect attribute reference is 
an attribute name prefixed by a \quote{C}-style \quote{*} dereference 
operator.

An attribute \emph{\bf reference} expects one attribute name or indirect 
reference, namely the attribute whose value, converted to a string using 
the lua \type{tostring} method, will be placed into the template 
\quote{hole} at that point. 

An attribute \emph{\bf application} expects a list of one or more attributes, 
the first of which denotes the template to which the rest of the 
attributes will be applied to.

An attribute \emph{\bf mapping} again expects a list of four or more 
attributes. The \emph{first attribute} refers to the attribute list which 
can be either a single value or an array of values whose individual items, 
in the case of an attribute array, are applied to the template identified 
by the third attribute. The \emph{second attribute} is the separator 
placed between subsequent expansions of the mapped template applied to 
each item of the attribute list. The \emph{third attribute} is a reference 
to a previously defined \quote{mapped template}. The \emph{fourth 
attribute} is the \quote{name}, internal to the mapped template given to 
each value of the attribute list. All \emph{subsequent attributes} are 
used as arguments applied to the mapped template. 

To render a template we will use a strategy of explicitly parsing all 
templates into a Lua structure which our rendering engine, below, can 
efficiently interpret.

\startTestSuite[prettyPrint]

While logically not required for our template engine, logistically, in 
order to \quote{debug} our templates, we need a way of \quote{tracing} the 
expansion of our templates. Of prime importance in the tracing of any 
template expansion is the ability to understand both the template and the 
environment used by that template at each recursive call. To do this, we 
start by providing a simple \quote{pretty print} method to provide a human 
readable rendition of an internal Lua data structure. 

Lua has the following list of value types: nil, boolean, number, string, 
function, userdata, thread, and table. Our pretty print method is a simple 
Lua \type{if}, \type{then}, \type{elseif}, and \type{else} structure. 
Tables are recursively pretty printed to provide a human readable way of 
understanding their internal structure. 

\startLuaCode
local function compareKeyValues(a, b)
  return (a[1] < b[1])
end

local function prettyPrint(anObj, indent)
  local result = ""
  indent = indent or ""
  if type(anObj) == 'nil' then
    result = 'nil'
  elseif type(anObj) == 'boolean' then
    if anObj then result = 'true' else result = 'false' end
  elseif type(anObj) == 'number' then
    result = toStr(anObj)
  elseif type(anObj) == 'string' then
    result = '"'..anObj..'"'
  elseif type(anObj) == 'function' then
    result = toStr(anObj)
  elseif type(anObj) == 'userdata' then
    result = toStr(anObj)
  elseif type(anObj) == 'thread' then
    result = toStr(anObj)
  elseif type(anObj) == 'table' then
    local origIndent = indent
    indent = indent..'  '
    result = '{\n'
    for i, aValue in ipairs(anObj) do
      result = result..indent..prettyPrint(aValue, indent)..',\n'
    end
    local theKeyValues = { }
    for aKey, aValue in pairs(anObj) do
      if type(aKey) ~= 'number' or aKey < 1 or #anObj < aKey then
        tInsert(theKeyValues,
          { prettyPrint(aKey), aKey, prettyPrint(aValue, indent) })
      end
    end
    tSort(theKeyValues, compareKeyValues)
    for i, aKeyValue in ipairs(theKeyValues) do
      result = result..indent..'['..aKeyValue[1]..'] = '..aKeyValue[3]..',\n'
    end
    result = result..origIndent..'}'
  else
    result = 'UNKNOWN TYPE: ['..toStr(anObj)..']'
  end
  return result
end

litProgs.prettyPrint = prettyPrint
\stopLuaCode

\startTestCase[test various examples]

\startLuaTest
local pp = thirddata.literateProgs.prettyPrint

assert.isEqual(pp(42),         '42')
assert.isEqual(pp('a string'), '"a string"')
assert.isEqual(pp(true),       'true')
assert.isEqual(pp(false),      'false')
assert.isEqual(pp({ 1, 2, 3}), '{\n  1,\n  2,\n  3,\n}')

assert.isEqual(pp({ x = 'x', y = 'y'}),
  '{\n  ["x"] = "x",\n  ["y"] = "y",\n}')
\stopLuaTest
\stopTestCase
\stopTestSuite

\startTestSuite[parseTemplate]

The \type{parseTemplate} method takes a string template and parses it into 
a Lua array of alternating string literals and attribute actions. This 
array will be walked over by the \type{renderer} rendering engine to 
product the final result of rendering the template. 

\startLuaCode
local function reportTemplateError(aTemplateStr, errMessage)
  texio.write_nl('---------------------------')
  texio.write_nl(errMessage)
  texio.write_nl("Template:")
  texio.write_nl(aTemplateStr)
  texio.write_nl("Ignoring this attribute action")
  texio.write_nl('---------------------------')
end

local function parseTemplate(aTemplateStr)
  local theTemplate = { }

  if type(aTemplateStr) == 'string' then
    -- we only do anything if we have been given
    -- an explicit string
    local position = 1
    while aTemplateStr:find('{{', position, true) do

      local endText, startChunk = aTemplateStr:find('{{', position, true)
      if position < endText then
        local textChunk = aTemplateStr:sub(position, endText-1)
        if textChunk and 0 < #textChunk then
          tInsert(theTemplate, textChunk)
        end
      end
      position = startChunk + 1

      local endChunk, startText = aTemplateStr:find('}}', position, true)
      if position < endChunk then
        local luaChunk = aTemplateStr:sub(position, endChunk-1)
        if luaChunk and 0 < #luaChunk then
          local actionType = luaChunk:sub(1,1)
          local luaChunk = luaChunk:sub(2)
          local arguments = { }
          for anArg in luaChunk:gmatch('[^,]+') do
            tInsert(arguments, anArg:match("^%s*(.-)%s*$"))
          end
          if actionType == '=' then
            if 0 < #arguments then
              tInsert(theTemplate, { 'reference', arguments[1] })
            else
              reportTemplateError(aTemplateStr,
                "No attribute specified"
              )
            end
          elseif actionType == '!' then
            if 0 < #arguments then
              tInsert(theTemplate, { 'application', arguments })
            else
              reportTemplateError(aTemplateStr,
                "No template specified"
              )
            end
          elseif actionType == '|' then
            if 2 < #arguments then
              tInsert(theTemplate, { 'mapping', arguments })
            else
              reportTemplateError(aTemplateStr,
                "No attribute, separator or template specified"
              )
            end
          else
            reportTemplateError(aTemplateStr,
              sFmt("Unknown template attribute action [%s]",
                actionType)
            )
          end
        end
      end
      position = startText + 1
    end

    if position <= #aTemplateStr then
      tInsert(theTemplate, aTemplateStr:sub(position))
    end
  end
  return theTemplate
end

litProgs.parseTemplate = parseTemplate
\stopLuaCode

\startTestCase[should parse simple templates]

\startLuaTest
local parseTemplate = thirddata.literateProgs.parseTemplate
local aTemplate = parseTemplate('This is a test')
assert.isTable(aTemplate)
assert.length(aTemplate, 1)
assert.isString(aTemplate[1])
assert.matches(aTemplate[1], 'This is a test')
--
aTemplate = parseTemplate(
  'This is a test [{{=  anAttribute  }}]'
)
assert.isTable(aTemplate)
assert.length(aTemplate, 3)
assert.isString(aTemplate[1])
assert.matches(aTemplate[1], 'This is a test %[')
assert.isTable(aTemplate[2])
assert.length(aTemplate[2], 2)
assert.matches(aTemplate[2][1], 'reference')
assert.matches(aTemplate[2][2], 'anAttribute')
assert.isString(aTemplate[3])
assert.matches(aTemplate[3], '%]')
--
aTemplate = parseTemplate(
  '{{| aList, aSeparator, aTemplate, anArgument}}'
)
assert.isTable(aTemplate)
assert.length(aTemplate,1)
assert.isTable(aTemplate[1])
assert.length(aTemplate[1], 2)
assert.matches(aTemplate[1][1], 'mapping')
assert.isTable(aTemplate[1][2])
assert.length(aTemplate[1][2], 4)
assert.matches(aTemplate[1][2][1], 'aList')
assert.matches(aTemplate[1][2][2], 'aSeparator')
assert.matches(aTemplate[1][2][3], 'aTemplate')
assert.matches(aTemplate[1][2][4], 'anArgument')
--
aTemplate = parseTemplate(
[=[
This is a test
but this {{= anAttribute }} is a reference
while this {{! aTemplate, anAttribute }} is an application
and this {{| aList, aSeparator, aTemplate, anArgument }}
is a mapping
]=])
assert.isTable(aTemplate)
assert.length(aTemplate,7)
assert.matches(aTemplate[1], 'This is a test\nbut this ')
assert.isTable(aTemplate[2])
assert.length(aTemplate[2], 2)
assert.matches(aTemplate[2][1], 'reference')
assert.matches(aTemplate[2][2], 'anAttribute')
assert.matches(aTemplate[3], ' is a reference\nwhile this ')
assert.isTable(aTemplate[4])
assert.matches(aTemplate[4][1], 'application')
assert.isTable(aTemplate[4][2])
assert.length(aTemplate[4][2], 2)
assert.matches(aTemplate[4][2][1], 'aTemplate')
assert.matches(aTemplate[4][2][2], 'anAttribute')
assert.matches(aTemplate[5], ' is an application\nand this ')
assert.isTable(aTemplate[6])
assert.matches(aTemplate[6][1], 'mapping')
assert.isTable(aTemplate[6][2])
assert.length(aTemplate[6][2], 4)
assert.matches(aTemplate[6][2][1], 'aList')
assert.matches(aTemplate[6][2][2], 'aSeparator')
assert.matches(aTemplate[6][2][3], 'aTemplate')
assert.matches(aTemplate[6][2][4], 'anArgument')
assert.matches(aTemplate[7], '\nis a mapping\n')
\stopLuaTest
\stopTestCase

\stopTestSuite

\startTestSuite[getReference]

The \type{getReference} method takes a given reference and looks it up in 
a given environment. Most importantly, the \type{getReference} method 
manages the recursive lookup of indirect references by stripping each 
successive \quote{C}-style \quote{*} dereference prefix operator and 
recursively calling the \type{getReference} method on the resulting 
reference. 

\startLuaCode
local function getReference(aReference, anEnv)
  if type(aReference) ~= 'string' then return nil end
  local redirect = aReference:sub(1,1)
  if redirect == '*' then
    local newRef = aReference:sub(2)
    local indirectRef = getReference(newRef, anEnv)
    if indirectRef and type(indirectRef) == 'string' then
      return getReference(indirectRef, anEnv)
    else
      return nil
    end
  end

  return anEnv[aReference]
end

litProgs.getReference = getReference
\stopLuaCode

\startTestCase[getReference should get indirect references]
\startLuaTest
local getReference = thirddata.literateProgs.getReference
local anEnv = { 
  anAttr = 'anAttrValue',
  attr   = 'anAttr'
}
assert.isNil(getReference(42))
assert.matches(getReference('anAttr', anEnv), 'anAttrValue')
assert.matches(getReference('attr',   anEnv), 'anAttr')
assert.matches(getReference('*attr',  anEnv), 'anAttrValue')
assert.isNil(  getReference('noAttr', anEnv))
\stopLuaTest
\stopTestCase
\stopTestSuite

\startTestSuite[parseTemplatePath]

To make it easy for humans to communicate, the \type{parseTemplatePath} 
method takes a simple \quote{.} separated list of Lua identifiers and 
parses it into a Lua ordered array of the individual Lua identifiers. This 
provides a simple namespace mechanism for organizing complex collections 
of templates. 

\startLuaCode
local function parseTemplatePath(templatePathStr, anEnv)
  if type(templatePathStr) ~= 'string' then
    texio.write_nl(
      sFmt("Expected [%s] to be a string.", tempaltePathStr)
    )
    texio.write_nl("Ignoring template.")
    return nil
  end
  if templatePathStr:sub(1,1) == '*' then
    templatePathStr = getReference(templatePathStr:sub(2), anEnv)
  end
  local templatePath = { }
  for subTemplate in templatePathStr:gmatch('[^%.]+') do
    tInsert(templatePath, subTemplate)
  end
  return templatePath
end

litProgs.parseTemplatePath = parseTemplatePath
\stopLuaCode

\startTestCase[parseTemplatePath should parse simple template paths]
\startLuaTest
local parseTemplatePath = thirddata.literateProgs.parseTemplatePath
local templatePath = parseTemplatePath('test.subTest.subSubTest')
assert.isTable(templatePath)
assert.length(templatePath, 3)
assert.matches(templatePath[1], 'test')
assert.matches(templatePath[2], 'subTest')
assert.matches(templatePath[3], 'subSubTest')
\stopLuaTest
\stopTestCase

\startTestCase[parseTemplatePath should parse indirect template references]

\startLuaTest
local parseTemplatePath = thirddata.literateProgs.parseTemplatePath
local anEnv = {
  indirectTemplatePath = 'test.subTest.subSubTest'
}
local templatePath = parseTemplatePath('*indirectTemplatePath', anEnv)
assert.isTable(templatePath)
assert.length(templatePath, 3)
assert.matches(templatePath[1], 'test')
assert.matches(templatePath[2], 'subTest')
assert.matches(templatePath[3], 'subSubTest')
\stopLuaTest
\stopTestCase
\stopTestSuite

\startTestSuite[navigateToTemplate]

The \type{navigateToTemplate} method takes a Lua ordered array of 
identifiers indicating a \quote{path} through the 
\type{thirddata.literateProgs.templates} table and returns the indicated 
template if it exists. 

\startLuaCode
local function navigateToTemplate(templatePath)
  litProgs.templates = litProgs.templates or { }
  local curTable = litProgs.templates
  for i, templateDir in ipairs(templatePath) do
    curTable[templateDir] = curTable[templateDir] or { }
    curTable = curTable[templateDir]
  end
  return curTable
end

litProgs.navigateToTemplate = navigateToTemplate
\stopLuaCode

\startTestCase[navigateToTemplate should navigate to subtables]
\startLuaTest
local parseTemplatePath = thirddata.literateProgs.parseTemplatePath
local templatePath = parseTemplatePath('test.subTest.subSubTest')
local navigateToTemplate = thirddata.literateProgs.navigateToTemplate
local curTable = navigateToTemplate(templatePath)
assert.isTable(curTable)
local templates = thirddata.templates
assert.isNotNil(templates)
assert.isTable(templates)
--
curtable = templates.test
assert.isNotNil(curTable)
assert.isTable(curTable)
--
curtable = curTable.subTest
assert.isNotNil(curTable)
assert.isTable(curTable)
--
curtable = curTable.subSubTest
assert.isNotNil(curTable)
assert.isTable(curTable)
--
templates.test = nil
assert.isNil(templates.test)
\stopLuaTest
\stopTestCase
\stopTestSuite

\startTestSuite[addTemplate]

The \type{addTemplate} method is used by any template file to manage the 
process of parsing a template into a structure located in the \ConTeXt\ 
\type{thirddata.literateProgs.templates} structure. The \type{addTemplate} 
method can be used as a Lua based DSL for managing templates. 

The \type{addTemplate} method expects three arguments, the 
\type{templatePath}, an ordered collection of the internal names of the 
formal arguments contained in \type{templateArgs}, and the 
\type{templateStr}. 

The \type{templatePath} must be a string containing a \quote{.} separated 
list of Lua identifiers. All templates are subtables of the 
\type{thirddata.literateProgs.templates} table. The \quote{.} separated 
list of lua identifiers provides a \quote{path} through these subtables to 
the ultimate template. This provides a simple namespace mechanism to help 
organize complex collections of templates. 

\startLuaCode
local function addTemplate(templatePathStr, templateArgs, templateStr)
  local templatePath = parseTemplatePath(templatePathStr)
  if not templatePath then return nil end
  local templateTable    = navigateToTemplate(templatePath)
  templateTable.path     = templatePathStr
  templateTable.args     = templateArgs
  templateTable.template = parseTemplate(templateStr)
end

litProgs.addTemplate = addTemplate
\stopLuaCode

\startTestCase[addTemplate should add a simple template]
\startLuaTest
local templates = thirddata.literateProgs.templates
templates.test = nil -- This should not be needed ;-(
assert.isNil(templates.test)
--
local addTemplate = thirddata.literateProgs.addTemplate
addTemplate('test.subTest.subSubTest',
  { 'arg1', 'arg2'},
  'This is a test template'
)
--
assert.isNotNil(templates.test)
assert.isNotNil(templates.test.subTest)
assert.isNotNil(templates.test.subTest.subSubTest)
local templateTable = templates.test.subTest.subSubTest
assert.isTable(templateTable)
assert.isNotNil(templateTable.args)
assert.isTable(templateTable.args)
assert.length(templateTable.args, 2)
assert.matches(templateTable.args[1], 'arg1')
assert.matches(templateTable.args[2], 'arg2')
assert.isNotNil(templateTable.template)
assert.isTable(templateTable.template)
assert.length(templateTable.template, 1)
assert.matches(templateTable.template[1], 'This is a test template')
--
templates.test = nil
assert.isNil(templates.test)
\stopLuaTest
\stopTestCase
\stopTestSuite

\startTestSuite[buildNewEnv]

On each template \quote{mapping} and \quote{application} action, the 
\type{renderer} rendering engine, defined below, recursively calls itself 
using the indicated template and an attribute \emph{environment}. The 
\type{buildNewEnv} method takes a template's formal arguments, a 
collection of actual arguments and an existing environment. It produces an 
new environment by mapping the values of actual arguments to their 
corresponding formal arguments. The correspondence is defined by the 
strict order of the formal and actual arguments. 

\startLuaCode
local function buildNewEnv(template, arguments, anEnv)
  if type(template)      ~= 'table' or
     type(template.args) ~= 'table' or
     type(arguments)     ~= 'table' or
     type(anEnv)         ~= 'table' then
    return { }
  end
  local formalArgs = template.args
  local newEnv = { }
  for i, aFormalArg in ipairs(formalArgs) do
    newEnv[aFormalArg] = getReference(arguments[i], anEnv)
  end
  return newEnv
end

litProgs.buildNewEnv = buildNewEnv
\stopLuaCode

\startTestCase[buildNewEnv should map formal to actual arguments]
\startLuaTest
local buildNewEnv = thirddata.literateProgs.buildNewEnv
local aTemplate = { 
  args = { 'argA', 'argB', 'argC' }, 
  template = 'aString'
}
local someArgs  = { 'attrA', 'attrB' }
local anEnv     = {
  attrA = 'attrAValue',
  attrB = 'attrBValue',
  attrC = 'attrCValue'
}
local newEnv = buildNewEnv(aTemplate, someArgs, anEnv)
assert.isTable(newEnv)
assert.matches(newEnv['argA'], 'attrAValue')
assert.matches(newEnv['argB'], 'attrBValue')
assert.isNil(  newEnv['argC'])
\stopLuaTest
\stopTestCase

\startTestCase[buildNewEnv should fail correctly]
\startLuaTest
local buildNewEnv = thirddata.literateProgs.buildNewEnv
assert.isTable(buildNewEnv('aString'))
assert.isTable(buildNewEnv( { }, 'aString'))
assert.isTable(buildNewEnv( { }, { }, 'aString'))
assert.isTable(buildNewEnv( { }, { }, { }))
assert.isTable(buildNewEnv( { { }}, { }, { }))
\stopLuaTest
\stopTestCase
\stopTestSuite

\startTestSuite[renderer]

Finally the \type{renderer} rendering engine is a \quote{template} 
interpreter. It walks over the template body interpreting each literal or 
action in turn. Any \emph{literal} is simply added as is to the result 
table. Any \emph{action} encountered is \quote{expanded} into a string and 
then added to the result table.

The expansion of a \emph{reference action} is simply the result of the Lua 
\type{tostring} method applied to the value returned by the 
\type{getReference} method defined above. 

An \emph{application action} is the result of recursively expanding the 
given template in the environment obtained by mapping the existing 
arguments to the template's formal arguments. 

A \emph{mapping action} is the result of repeatedly expanding the given 
template in the environment obtained by mapping the existing arguments to 
the template's formal arguments. The indicated separator will be placed 
\emph{between} each expansion. 

\startLuaCode
local function renderer(aTemplate, anEnv)
  if type(aTemplate) == 'table' and
     type(aTemplate.template) == 'table' then
    local result = { }
    for i, aChunk in ipairs(aTemplate.template) do
      if type(aChunk) == 'string' then
        -- a simple litteral string... add it
        tInsert(result, aChunk)
      elseif type(aChunk) == 'table' and 2 <= #aChunk then
        local actionType = aChunk[1]
        local arguments  = aChunk[2]
        if actionType == 'reference' then
          if type(arguments) == 'string' then
            local attrValue = getReference(arguments, anEnv)
            tInsert(result, toStr(attrValue))
          end
        elseif actionType == 'application' then
          if type(arguments) == 'table' and 1 < #arguments then
            local templatePath = tRemove(arguments, 1)
            if type(templatePath) == 'string' then
              templatePath   = parseTemplatePath(templatePath, anEnv)
              local template = navigateToTemplate(templatePath)
              local newEnv   = buildNewEnv(template, arguments, anEnv)
              local templateValue = renderer(template, newEnv)
              if type(templateValue) == 'string' then
                tInsert(result, templateValue)
              end
            end
          end
        elseif actionType == 'mapping' then
          if type(arguments) == 'table' and 4 <= #arguments then
            local attrList     = tRemove(arguments, 1)
            local separator    = tRemove(arguments, 1)
            local templatePath = tRemove(arguments, 1)
            local listArg      = tRemove(arguments, 1)
            if type(attrList)     == 'string' and
               type(separator)    == 'string' and
               type(templatePath) == 'string' and
               type(listArg)      == 'string' then
              attrList       = getReference(attrList,  anEnv)
              separator      = getReference(separator, anEnv)
              templatePath   = parseTemplatePath(templatePath, anEnv)
              local template = navigateToTemplate(templatePath)
              local newEnv   = buildNewEnv(template, arguments, anEnv )
              if type(separator) ~= 'string' then
                separator = ''
              end
              if type(attrList) ~= 'table' then
                attrList = { attrList }
              end
              if type(attrList) == 'table' and 0 < #attrList then
                for i, anAttrValue in ipairs(attrList) do
                  newEnv[listArg] = anAttrValue
                  local templateValue = renderer(template, newEnv)
                  if type(templateValue) == 'string' then
                    tInsert(result, templateValue)
                  end
                  if i < #attrList then
                    tInsert(result, separator)
                  end
                end
              end
            end
          end
        end
      end
    end
    return tConcat(result)
  else
    return ""
  end
end

litProgs.renderer = renderer
\stopLuaCode

\startTestCase[renderer should ...]
\startLuaTest
local templates   = thirddata.literateProgs.templates
local renderer    = thirddata.literateProgs.renderer
local addTemplate = thirddata.literateProgs.addTemplate
addTemplate( 
  'test.templateA',
  { 'anArg' },
  'anArg == [{{= anArg}}]'
)
assert.isTable(templates.test)
assert.isTable(templates.test.templateA)
addTemplate(
  'test.templateB',
  { 'anArg' },
[=[
This is a test {{= anArgA }}
so is this {{! test.templateA, anArgB }}
and this {{| argList, newLines, test.templateA, anArg }}
]=]
)
assert.isTable(templates.test.templateB)
local anEnv = {
  argList = { 1, 2, 3 },
  newLines = ',\n',
  anArgA = 'argAValue',
  anArgB = 'argBValue'
}
local templateB = templates.test.templateB
local result = renderer(templateB, anEnv)
assert.matches(result, 'test argAValue\n')
assert.matches(result, 'this anArg == %[argBValue%]\n')
assert.matches(result, 'this anArg == %[1%],\n')
assert.matches(result, 'anArg == %[3%]\n')
\stopLuaTest
\stopTestCase
\stopTestSuite

Lastly we define a simple \type{renderCodeFile} method to write out a 
given code table into the contents of the given file. 

\startLuaCode
local function renderCodeFile(aFilePath, codeTable)
  local outFile = io.open(aFilePath, 'w')
  outFile:write(tConcat(codeTable, '\n\n'))
  outFile:close()
end
\stopLuaCode

\stopchapter