% A ConTeXt document [master document: literateProgs.tex]

\startchapter[title=The Lua Rendering Engine]

We follow Terance Parr's advice, \cite{parr2004templateMVC}, to ensure 
there is a clear separation between Model and View when rendering 
templates. In his view, the rendering process is a form of 
\quote{unparsing} or, alternatively, language generation. Using Parr's 
\quote{push} semantics, a template is defined to be a litteral string with 
embedded pairs of \quote{\{\{}, \quote{\}\}} delimited \quote{holes} or 
attribute actions, where each attribute action must not have side-effects 
on the data model. 

Our template engine will support two semantic actions: simple attribute 
reference and mapping a named template against a list of attributes. To do 
this we will have three forms of attribute actions: 

\startitemize
\item {\bf reference}: \type{{{= <attribute-list> }}} 

\item {\bf application}: \type{{{! <attribute-list> }}}

\item {\bf mapping}: \type{{{| <attribute-list> }}}

\stopitemize

Note that attribute application is simply a convenient form of attribute 
mapping. Note as well that while attribute mapping is a form of \type{for} 
\type{in} looping, it, and attribute application, can be used to replace 
any need for explicit \type{if}, \type{elseif}, \type{else}, \type{endif} 
structures. 

In each case, the \type{<attribute-list>} is a comma separated list of 
attribute names or indirect references. An indirect attribute reference is 
an attribute name enclosed in \quote{\[} \quote{\]} pairs. An attribute 
\emph{reference} expects one attribute name or indirect reference, namely 
the attribute whose value will be placed into the template \quote{hole} at 
that point. An attribute \emph{application} expects a list of one or more 
attributes, the first of which denotes the template to which the rest of 
the attributes will be applied to. An attribute \emph{mapping} again 
expects a list of two or more attributes. The first attribute refers to 
the attribute or attribute array whose individual items, in the case of an 
attribute array, are applied to the template identified by the second 
attribute. All subsequent attributes are used as arguments applied to the 
template. 

To render a template we will use a strategy of explicitly parsing all 
templates into a Lua structure which our rendering engine, below, can 
efficiently interpret.

\startTestSuite[parseTemplate]

The \type{parseTemplate} method takes a string template and parses it into 
a Lua array of alternating string literals and attribute actions. This 
array will be walked over by the \type{renderer} rendering engine to 
product the final result of rendering the template. 

\startLuaCode
function litProgs.parseTemplate(aTemplateStr)
  local theTemplate = { }

  if type(aTemplateStr) == 'string' then
    -- we only do anything if we have been given
    -- an explicit string 
    local position = 1
    while aTemplateStr:find('{{', position) do
    
      local textChunk = aTemplateStr:match('^.*{{', position)
      if textChunk then 
        local textChunkLen = #textChunk
        textChunk = textChunk:sub(1, textChunkLen-2)
        if 0 < #textChunk then tInsert(theTemplate, textChunk) end
        position = position + textChunkLen
      end
      
      local luaChunk = curTemplate:match('^.+}}', position)
      if luaChunk then
        local luaChunkLen = #luaChunk
        luaChunk = luaChunk:sub(1, luaChunkLen-2)
        if 0 < #luaChunk then
          texio.write_nl('['..luaChunk:sub(1,1)..']')
          tInsert(theTemplate, luaChunk)
        end
        position = position + luaChunkLen
      end
    end
    
    if position < #aTemplateStr then
      tInsert(theTemplate, aTemplateStr:sub(position))
    end
  end
  return theTemplate
end
\stopLuaCode

\startTestCase[should parse simple templates]

\startLuaTest
local parseTemplate = thirddata.literateProgs.parseTemplate
local aTemplate = parseTemplate('This is a test')
assert.isTable(aTemplate)
assert.length(aTemplate, 1)
assert.isString(aTemplate[1])
assert.matches(aTemplate[1], 'This is a test')
--
aTemplate = parseTemplate(
  'This is a test{{=  anAttribute  }}This is another test'
)
assert.isTable(aTemplate)
assert.length(aTemplate, 3)
assert.isString(aTemplate[1])
assert.matches(aTemplate[1], 'This is a test')
assert.isTable(aTemplate[2])
assert.length(aTemplate[2], 2)
assert.matches(aTemplate[2][1], 'reference')
assert.matches(aTemplate[2][2], 'anAttribute')
assert.isString(aTemplate[3])
assert.matches(aTemplate[3], 'This is another test')
--
aTemplate = parseTemplate(
  '{{| aList, aSeparator, aTemplate, anArgument}}'
)
assert.isTable(aTemplate)
assert.length(aTemplate,1)
assert.isTable(aTemplate[1])
assert.length(aTemplate[1], 2)
assert.matches(aTemplate[1][1], 'mapping')
assert.isTable(aTemplate[1][2])
assert.length(aTemplate[1][2], 4)
assert.matches(aTemplate[1][2][1], 'aList')
assert.matches(aTemplate[1][2][2], 'aSeparator')
assert.matches(aTemplate[1][2][3], 'aTemplate')
assert.matches(aTemplate[1][2][4], 'anArgument')
--
aTemplate = parseTemplate(
[=[
This is a test
but this {{= anAttribute }} is a reference
while this {{! aTemplate, anAttribute }} is an application
and this {{| aList, aSeparator, aTemplate, anArgument }}
is a mapping
]=])
assert.isTable(aTemplate)
showValue(aTemplate)
assert.length(aTemplate,7)
assert.matches(aTemplate[1], 'This is a test\nbut this ')
assert.isTable(aTemplate[2])
assert.length(aTemplate[2], 2)
assert.matches(aTemplate[2][1], 'reference')
assert.matches(aTemplate[2][2], 'anAttribute')
assert.matches(aTemplate[3], ' is a reference\nwhile this ')
assert.isTable(aTemplate[4])
assert.matches(aTemplate[4][1], 'application')
assert.isTable(aTemplate[4][2])
assert.length(aTemplate[4][2], 2)
assert.matches(aTemplate[4][2][1], 'aTemplate')
assert.matches(aTemplate[4][2][2], 'anAttribute')
assert.matches(aTemplate[5], ' is an application\nand this ')
assert.isTable(aTemplate[6])
assert.matches(aTemplate[6][1], 'mapping')
assert.isTable(aTemplate[6][2])
assert.length(aTemplate[6][2], 4)
assert.matches(aTemplate[6][2][1], 'aList')
assert.matches(aTemplate[6][2][2], 'aSeparator')
assert.matches(aTemplate[6][2][3], 'aTemplate')
assert.matches(aTemplate[6][2][4], 'anArgument')
assert.matches(aTemplate[7], '\nis a mapping\n')
\stopLuaTest
\showLuaTest
\stopTestCase

\stopTestSuite

The \type{addTemplate} method is used by any template files to manage the 
process of parsing a template into a structure located in the \ConTeXt\ 
\type{thirddata.templates} structure. The \type{addTemplate} method can be 
used as a Lua based DSL for managing templates. 

\startLuaCode
function litProgs.addTemplate()

end
\stopLuaCode

\startLuaCode
function litProgs.renderer()

end
\stopLuaCode

What follows is our previous attempt at rendering a template. This will be 
phased out once the new type of template rendering is fully functional. 

\startLuaCode

-- We need a simple Lua based template engine
-- Our template engine has been inspired by:
--   https://john.nachtimwald.com/2014/08/06/using-lua-as-a-templating-engine/
-- (via the minLua JoyLoL template engine)

function litProgs.renderNextChunk(prevChunk, renderedText, curTemplate)
  local result = ""
  
  if prevChunk
    and type(prevChunk) == 'string'
    and 0 < #prevChunk then
    tInsert(renderedText, prevChunk)
  end
  
  if type(curTemplate) == 'string' and (0 < #curTemplate) then
    if curTemplate:find('{{') then
      local position  = 1
      local textChunk = curTemplate:match('^.*{{', position)
      if textChunk then 
        local textChunkLen = #textChunk
        textChunk = textChunk:sub(1, textChunkLen-2)
        if 0 < #textChunk then tInsert(renderedText, textChunk) end
        position = position + textChunkLen
      end
      
      local luaChunk = curTemplate:match('^.+}}', position)
      if luaChunk then
        local luaChunkLen = #luaChunk
        luaChunk = luaChunk:sub(1, luaChunkLen-2)
        position = position + luaChunkLen
        curTemplate = curTemplate:sub(position, #curTemplate)
        local newChunk = ""
        if not luaChunk:match('^%s*$') then
          -- consider using an PCall here....
          local luaFunc, errMessage = load(luaChunk)
          if luaFunc then
            newChunk = luaFunc(litProgs)
          end
        end
        result = litProgs.renderNextChunk(newChunk, renderedText, curTemplate)
      end
    else -- there is no '{{' in the template
      tInsert(renderedText, curTemplate)
      result = tConcat(renderedText)
    end
  else
    -- nothing to do...
    result = tConcat(renderedText)
  end
  return result
end

function litProgs.render(aTemplate)
  return litProgs.renderNextChunk("", { }, aTemplate)
end

-- Now we need the code that captures and creates a given code/file type 

local function renderFile(aFilePath, baseTemplate)
  local outFile = io.open(aFilePath, 'w')
  --outFile:write(pp.write(litProgs))
  local renderedBaseTemplate = litProgs.renderNextChunk("", {}, baseTemplate)
  --outFile:write('\n--------------\n')
  outFile:write(renderedBaseTemplate)
  outFile:close()
end

\stopLuaCode

\section[title=Testing the rendering engine]

Given the moderate complexity of the Lua rendering engine, we need to 
provide justifications of its correctness. While Theoretical Computer 
Science does provide tools which could be used to attempt to \emph{prove} 
the above Lua code correct these tools are not really integral to the Lua 
language. Our primary aim in creating this cycle of tools is to develop 
the JoyLoL language in which these tools of \emph{deductive} proof are 
integral to the language. In the JoyLoL language, the distinction between 
deductive proof and inductive testing is much clearer. For the moment we 
will only provide \quote{inductive} tests of correctness for the above Lua 
code. 

\startTestSuite[renderNextChunk]

At the moment, the critical complexity is contained in the 
\type{renderNextChunk} function, so we will focus upon testing this 
function. Our strategy will be simple, we will invoke the 
\type{renderNextChunk} function on a number of example arguments, chosen 
to ensure all of the \quote{main} paths of the function are taken in at 
least one of the invocations. Associated with each invocation, we will 
make a number of simple assertions about that invocation’s returned value. 
If all of the assertions are true, we consider the \type{renderNextChunk} 
function \quote{inductively} correct. 

\startTestCase[should stop]

The \type{renderNextChunk} function is recursive, so in our first test 
case, it is important to test that this recursion \emph{does} stop. To do 
this we provide arguments which have a simple previous chunk 
(\type{prevChunk}), a simple previously rendered text 
(\type{renderedText}), as well as an \emph{empty} currently remaining 
template (\type{curTemplate}). 

\startLuaTest
  local result = thirddata.literateProgs.renderNextChunk(
    "this is a simple text", { }, ""
  )
\stopLuaTest

The result of this invocation, should be a Lua string which consists 
of the \type{preChunk}. 

\startLuaTest
  assert.isString(result)
  assert.isEqual("this is a simple text", result)
\stopLuaTest
\stopTestCase

\startTestCase[should combine previously rendered text]

We next need to show that more complex collections of previously rendered 
text are handled correctly. 

\startLuaTest
  local result = thirddata.literateProgs.renderNextChunk(
    "this is a final text",
    {
      "this is the first text",
      "this is the second text"
    },
    ""
  )
\stopLuaTest

The result should be the texts contained in the rendered text together 
with the \type{prevChunck} all concatenated \emph{with no additional 
white-space}. 

\startLuaTest
  assert.isString(result)
  assert.isEqual(
    "this is the first textthis is the second textthis is a final text",
    result
  )
\stopLuaTest
\stopTestCase

\stopTestSuite

\stopchapter