% A ConTeXt document [master document: literateProgs.tex]

\startchapter[title=MkIV code]

Any \ConTeXt\ module consists of two distinct types of code which can be 
broken into multiple distinct files. Every module contains \ConTeXt\ 
\type{MkIV} code, which is basically plain \TeX\ macros together with a 
collection of \ConTeXt\ specific \quote{helper} macros which together 
provide consistent interfaces for end-users. 

In this and the next chapter, we cover the description and testing of the 
\type{MkIV} code. 

As mentioned previously, we use the standard \ConTeXt\ \type{typing} 
environment to both capture and typeset the \type{MkIV} code. 

To use the \type{typing} environment, we invoke the \ConTeXt\ 
\type{\definetyping} and \type{\setuptyping} macros with the name of the 
desired \type{typing} environment, in our case \type{MkIVCode}, as well as 
any particular default values for the environment's setup. In our case 
this creates the \type{\startMkIVCode} and \type{\stopMkIVCode} macros 
together with the whole \type{\setupMkIVCode} interface tailored to our 
specific name. 

If we only cared about the \emph{typesetting} of the MkIVCode, this is all 
we need to do, as \ConTeXt\ is already competent at typesetting \TeX\ and 
Lua code fragments. However, we \emph{also} want to store the code chunk 
in Lua tables for further rendering to an executable file format. To do 
this we need to redefine the \type{\stopMkIVCode} macro by adding a call 
to our Lua function \type{addMkIVCode}, while at the same time invoking 
all of the standard \TeX\ macros associated with the \quote{original} 
\type{\stopMkIVCode} macro. We do this by \type{\let}ing 
\type{\oldStopMkIVCode} be the original macro \type{\stopMkIVCode} and 
then redefining \type{\stopMkIVCode} to be an invokation of the original 
\type{\stopMkIVCode} followed by our call to the Lua function 
\type{addMkIVCode}. 

The \type{typing} environment of necessity changes \TeX's cat-codes in 
such a way that \TeX\ respects all white-space. This allows the 
\type{typing} environment to capture the complete verbatim layout of the 
contents of the \type{typing} environment. However it also means that, in 
our case, the implementation actively scans for the particular 
\emph{string} \type{\stopMkIVCode}.

Since, in our literate modules approach to the creation of \ConTeXt\ 
modules, we are \emph{defining} the \type{\stopMkIVCode} macro using the 
\quote{MkIVCode} \type{typing} environment itself, we seriously confuse 
\ConTeXt's \type{typing} environment's search for the \emph{string} 
\type{\stopMkIVCode}. This is because we need to embed an instance of the 
string \type{\stopMkIVCode} inside the \type{typing} environment. To avoid 
this confustion, we actually define \emph{two} \type{MkIVCode} 
environments, \type{MkIVCode} (itself) as well as \type{MkIVAltCode}. This 
allows use to define each environment inside the other environment. 

We can now define the \type{\stopMkIVCode} macro using the 
\type{MkIVAltCode} environment. 

Having defined the \type{\stopMkIVCode}, we can now define the 
\type{\stopMkIVAltCode} macro using the \type{MkIVCode} environment. 

 Once the \type{MkIV} code has been captured and typeset, we need to 
specify where we want to store it until we ultimately render it via 
templates into the corresponding \type{MkIV} files. To do this we define 
the \type{addMkIVCode} function in Lua, inside the \type{litProgs} table. 
The \type{addMkIVCode} function is invoked by the \type{\stopMkIVCode} 
macro and takes the \ConTeXt\ buffer name, in this case always 
\quote{_typing_}, from which to extract the last \quote{chunk} of 
\type{MkIV} code. The \type{addMkIVCode} takes the contents of the buffer 
and inserts it at the end of the \type{code.mkiv} array. 

\startLuaCode
local function markMkIVCodeOrigin(fileName, startLine)
  tInsert(code.mkiv,
    sFmt('%% from file: %s after line: %s',
      fileName,
      toStr(
        mFloor(
          startLine/code.lineModulus
        )*code.lineModulus
      )
    )
  )
end

litProgs.markMkIVCodeOrigin = markMkIVCodeOrigin
\stopLuaCode

 
To actually initiate the rendering of the \type{MkIV} code into a specific 
file, the module writer actively invokes the \type{\createMkIVFile} macro, 
and supplies the output file's relative path and file name complete with 
extension. 

The \type{\createMkIVFile} macro in turn invokes the \type{createMkIVFile} 
Lua function. 

\stopchapter