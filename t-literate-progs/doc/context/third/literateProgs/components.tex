% A ConTeXt document [master document literateProgs.tex]

\chapter[title=Working with ConTeXt Components]

We start by determining whether or not we want to generate the whole 
typeset document or just the auto-generated code. We do this by asserting 
the \ConTeXt\ \quote{mode} \quote{codeOnly}. This can be set in the 
document, before any line containing \type{\usemodule[t-litProgs]}, by using 
the command, \type{\enablemode[codeOnly]}, to ensure only the \ConTeXt\ 
document is only run in order to generate the auto-generated code. 
Alternatively you can use the command, \type{\disablemode[codeOnly]}, to 
ensure the whole document is type set. 

\startMkIVCode
\doifmodeelse{codeOnly}{
  \directlua{
    thirddata.literateProgs.setCodeOnly(true)
  }
} {
  \directlua{
    thirddata.literateProgs.setCodeOnly(false)
  }
}

\stopMkIVCode

\startLuaCode
local function setCodeOnly(codeOnlyBoolean)
  if codeOnlyBoolean then
    texio.write_nl('setCodeOnly to true')
  else
    texio.write_nl('setCodeOnly to false')
  end
  litProgs.codeOnly = codeOnlyBoolean
  if litProgs.codeOnly then
    texio.write_nl('setCodeOnly now true')
  else
    texio.write_nl('setCodeOnly now false')
  end
end

litProgs.setCodeOnly = setCodeOnly
\stopLuaCode

\section[title=LitProgslex document Environments and Components macros]

We specialize the standard \type{\environment} and \type{\component} macros 
to ensure two things:

\startitemize[n]

\item \bold{complex collections of sub-components found in \quote{relative} 
paths}. LitProgs documents may be built using multiple associated artifacts 
such as, for example, CoAlgebras. We need the ability to specify 
components which might not be in the traditional \ConTeXt\ hierarchical 
directory structures.

\item \bold{knowledge of \quote{doc} directories}. LitProgs documents usually 
produce computational artifacts which might need separate compilation. To 
keep these \emph{derived} computational artefacts separate from the main 
document, we keep \ConTeXt\ files in \quote{doc} directories and any 
derived computational artefacts in \quote{buildDir} directories.

\stopitemize

We implement \type{\startLitProgsComponent} and \type{\stopLitProgsComponent} 
in order to ensure that our complex collection of nested components does 
not break \ConTeXt's native \quote{project}, \quote{product}, and 
\quote{component} system. We effectively only make use of 
\quote{components}, but we explicitly allow them to be nested multiple 
levels deep. To do this we only actually let \ConTeXt\ \quote{see} the 
outer most start/stop \quote{component} declaration.

\startMkIVCode

%% repeat after me.... this WILL break!!!
%%
%% this call pattern has been modeled upon the
%% definition of \environment in file-job.mkvi
%% on 2018/Nov/07 (experimental distribution)
%%
\unexpanded\def\litProgsEnvironment{
  \doifelsenextoptionalcs%
    \useLitProgsEnvironment%
    \syst_structure_arg_litprogs_environment%
}

\def\syst_structure_arg_litprogs_environment  #1 {\useLitProgsEnvironment  [#1]}

\unexpanded\def\useLitProgsEnvironment  [#1]{
  \directlua{
    thirddata.literateProgs.litProgsComponent('environment', '#1')
  }
}

%% repeat after me.... this WILL break!!!
%%
%% this call pattern has been modeled upon the
%% definition of \startcomponent in file-job.mkvi
%% on 2018/Nov/07 (experimental distribution)
%%
\unexpanded\def\startLitProgsComponent{
  \doifelsenextoptionalcs%
    \useStartLitProgsComponent%
    \syst_structure_arg_start_litprogs_component%
}

\def\syst_structure_arg_start_litprogs_component  #1 {\useStartLitProgsComponent[#1]}

\unexpanded\def\useStartLitProgsComponent  [#1]{
  \directlua{
    thirddata.literateProgs.startLitProgsComponent('component', '#1')
  }
}

\def\stopLitProgsComponent{
  \directlua{
    thirddata.literateProgs.stopLitProgsComponent('component')
  }
}

%% repeat after me.... this WILL break!!!
%%
%% this call pattern has been modeled upon the
%% definition of \component in file-job.mkvi
%% on 2018/Nov/07 (experimental distribution)
%%
\unexpanded\def\litProgsComponent{
  \doifelsenextoptionalcs%
    \useLitProgsComponent%
    \syst_structure_arg_litprogs_component%
}

\def\syst_structure_arg_litprogs_component  #1 {  \useLitProgsComponent  [#1]}

\unexpanded\def\useLitProgsComponent  [#1]{
  \directlua{
    thirddata.literateProgs.litProgsComponent('component', '#1')
  }
}

\def\popLitProgsPath{
  \directlua{
    thirddata.literateProgs.popLitProgsPath()
  }
}

\def\currentLitProgsPath{
  \directlua{
    tex.print(thirddata.literateProgs.lastLitProgsPath())
  }
}

\def\showLitProgsMessages{
  \directlua{
    thirddata.literateProgs.showLitProgsMessages(true)
  }
}

\def\hideLitProgsMessages{
  \directlua{
    thirddata.literateProgs.showLitProgsMessages(false)
  }
}
\stopMkIVCode

\startLuaCode
local showMessages = false

local function showLitProgsMessages(aShowMessages)
  showMessages = aShowMessages
end

litProgs.showLitProgsMessages = showLitProgsMessages

local insideComponent = {}
insideComponent['component']   = 0
insideComponent['environment'] = 0
insideComponent['product']     = 0
insideComponent['project']     = 0

local litProgsPaths   = {}
local pathSeparator = package.config:sub(1, 1)

local function lastLitProgsPath()
  return litProgsPaths[#litProgsPaths] or ""
end

litProgs.lastLitProgsPath = lastLitProgsPath

local function pushLitProgsPath(aFullPath)
  if showMessages then
    texio.write_nl('pushLitProgsPath('..aFullPath..')')
  end
  local aFullPathDir =
    aFullPath:gsub('[^'..pathSeparator..']+$', '')
  if showMessages then
    texio.write_nl('  aFullPathDir: ['..aFullPathDir..']')
  end
  if aFullPathDir:sub(-1) ~= '/' then
    aFullPathDir = aFullPathDir..pathSeparator
  end
  tInsert(litProgsPaths, aFullPathDir)
  if showMessages then
    for i, aPath in ipairs(litProgsPaths) do
      texio.write_nl('litProgsPaths['..toStr(i)..']: ['..aPath..']')
    end
  end
end

-- repeat after me... this WILL break!!!
--
-- the use of environments.arguments.fulljobname
-- was infered from grep'ing the experimental distribution
-- for fulljobname and finding it defined in the
-- the environment table. 
-- (defined in core-sys.lua)
--
pushLitProgsPath(file.collapsepath(environment.arguments.fulljobname,true))

local function popLitProgsPath()
  if showMessages then
    texio.write_nl('popLitProgsPath()')
  end
  tRemove(litProgsPaths)
  if showMessages then
    for i, aPath in ipairs(litProgsPaths) do
      texio.write_nl('litProgsPaths['..toStr(i)..']: ['..aPath..']')
    end
  end
  if showMessages then
    texio.write_nl('<<<<<<<<<<<<<<<<<<<<<<<<<<<<<<<<<<<<<<<<<<<<<<<<<<<<<')
  end
end

litProgs.popLitProgsPath = popLitProgsPath

local function findLitProgsPath(curBasePath, componentPath, origBasePath)
  if showMessages then
    texio.write_nl('findLitProgsPath(['..curBasePath..'],['..componentPath..'],['..origBasePath..'])')
  end
  local potentialPath =
    file.collapsepath(curBasePath..componentPath, true)
  if lfs.attributes(potentialPath..'.tex', 'mode') == 'file' then
    if showMessages then
      texio.write_nl('found: ['..potentialPath..']')
    end
    return potentialPath
  end
  potentialPath =
    file.collapsepath(curBasePath..'doc/'..componentPath, true)
  if lfs.attributes(potentialPath..'.tex', 'mode') == 'file' then
    if showMessages then
      texio.write_nl('found: ['..potentialPath..']')
    end
    return potentialPath
  end
  if curBasePath == '' or curBasePath == pathSeparator then
    texio.write_nl('no path found using: ['..origBasePath..componentPath..']')
    return file.collapsepath(origBasePath..componentPath, true)
  end
  local newCurBasePath =
    curBasePath:gsub('[^'..pathSeparator..']+'..pathSeparator..'$','')
  return findLitProgsPath(newCurBasePath, componentPath, origBasePath)
end

local function litProgsComponent(componentType, componentPath)
  if showMessages then
    texio.write_nl('>>>>>>>>>>>>>>>>>>>>>>>>>>>>>>>>>>>>>>>>>>>>>>>>>>>>>')
    texio.write_nl('litProgsComponent(['..componentType..'],['..componentPath..'])')
  end
  if componentType == 'environment' and 1 < insideComponent['component'] then
    -- do nothing
  else
    -- we are either NOT an environment
    -- OR we are an environment but inside the first start/stop component pair
    local basePath = lastLitProgsPath()
    local thisComponentPath = findLitProgsPath(basePath, componentPath, basePath)
    if showMessages then
      texio.write_nl(' thisComponentPath: ['..thisComponentPath..']')
    end
    pushLitProgsPath(thisComponentPath)
    tex.print({
      '\\'..componentType..' '..thisComponentPath,
      '\\popLitProgsPath'
    })
  end
end

litProgs.litProgsComponent = litProgsComponent

local function startLitProgsComponent(componentType, componentName)
  if showMessages then
    texio.write_nl('startLitProgsComponent(['..componentType..'],['..componentName..']')
    for k,v in pairs(insideComponent) do
      texio.write_nl('insideComponent['..k..'] = '..toStr(v))
    end
  end
  if insideComponent[componentType] < 1 then
    tex.print('\\start'..componentType..' '..componentName..'\\relax')
  end
  insideComponent[componentType] = insideComponent[componentType] + 1
  if showMessages then
    texio.write_nl(
      '\\startLitProgsComponent('..componentType..')'..
      '['..toStr(insideComponent[componentType])..']'
    )
  end
end

litProgs.startLitProgsComponent = startLitProgsComponent

local function stopLitProgsComponent(componentType)
  if showMessages then
    texio.write_nl('stopLitProgsComponent(['..componentType..']')
    texio.write_nl(
      '\\stopLitProgsComponent('..componentType..')'..
      '['..toStr(insideComponent[componentType])..']'
    )
  end
  insideComponent[componentType] = insideComponent[componentType] - 1
  if showMessages then
    for k,v in pairs(insideComponent) do
      texio.write_nl('insideComponent['..k..'] = '..toStr(v))
    end
  end
  if insideComponent[componentType] < 1 then
    if insideComponent[componentType] < 0 then
      texio.write_nl('ERRROR: unbalanced number of \\stop'..componentType)
    end
    if showMessages then
      texio.write_nl('CALLING \\stop'..componentType)
    end
    tex.print('\\stop'..componentType..'\\relax')
  end
end

litProgs.stopLitProgsComponent = stopLitProgsComponent

\stopLuaCode

Now we deal with recursive descent into sub-documents.

\startMkIVCode
\def\litProgsRecurseComponent#1#2#3{
  \directlua{
    thirddata.literateProgs.litProgsRecurseComponent('component', '#1', '#2', '#3')
  }
}
\stopMkIVCode

\startLuaCode
local function litProgsRecurseComponent(componentType, basePath, docDir, componentName)
  if litProgs and litProgs.addLmsfileSubDirectory then
    litProgs.addLmsfileSubDirectory(basePath)
  end
  if not litProgs.codeOnly then
    litProgsComponent(
      componentType,
      makePath{ '..', basePath, docDir, componentName}
    )
  end
end

litProgs.litProgsRecurseComponent = litProgsRecurseComponent
\stopLuaCode