% A ConTeXt document [master document: literateProgs.tex]

\startchapter[title=Preamble]

Any description of almost any code artefact, begins with a preamble. The 
preamble sets the stage for the rest of the code. In our case, the 
respective preambles provide:
%
\startitemize
\item a machine readable description of the file,
\item a human readable copyright,
\item commands to use or include other artefacts
\item command to setup the correct internal state
\stopitemize

\defineLitProgs[MkIVCode][option=context]
\defineLitProgs[MpIVCode][option=mp]
\defineLitProgs[LuaCode][option=lua]
\defineLitProgs[LuaTemplate][option=lua]

\addMITLicense{MkIVCode}{}{PerceptiSys Ltd (Stephen Gaito)}

\startMkIVCode
%D \module
%D   [     file=t-literate-progs,
%D      version=2017.05.10,
%D        title=\CONTEXT\ User module,
%D     subtitle=Literate Programming in \ConTeXt,
%D       author=Stephen Gaito,
%D         date=\currentdate,
%D    copyright=PerceptiSys Ltd (Stephen Gaito),
%D        email=stephen@perceptisys.co.uk,
%D      license=MIT License]

% begin info
%
% title   : Literate Programming in ConTeXt
% comment : Provides structured document and code generation
% status  : under development, mkiv only
%
% end info

\usemodule[high-cpp]

\unprotect

\ctxloadluafile{t-literate-progs}
\ctxloadluafile{t-literate-progs-templates}
\stopMkIVCode

\section{Lua code}

\addMITLicense{LuaCode}{--}{PerceptiSys Ltd (Stephen Gaito)}

\startLuaCode
-- This is the lua code associated with t-literate-progs.mkiv

if not modules then modules = { } end
modules ['t-literate-progs'] = {
    version   = 1.000,
    comment   = "Literate Programming in ConTeXt - lua",
    author    = "PerceptiSys Ltd (Stephen Gaito)",
    copyright = "PerceptiSys Ltd (Stephen Gaito)",
    license   = "MIT License"
}

thirddata               = thirddata               or {}
thirddata.literateProgs = thirddata.literateProgs or {}

local litProgs   = thirddata.literateProgs
litProgs.code    = litProgs.code or {}
local code       = litProgs.code
code.mkiv        = {}
code.mpiv        = {}
code.lua         = {}
code.templates   = {}
code.lmsfile     = {}
code.lineModulus = 50
litProgs.build   = litProgs.build or {}
local build      = litProgs.build

local tInsert = table.insert
local tRemove = table.remove
local tConcat = table.concat
local tSort   = table.sort
local tUnpack = table.unpack
local sFmt    = string.format
local sMatch  = string.match
local mFloor  = math.floor
local toStr   = tostring
\stopLuaCode

\startTestSuite[setDefs]

\startLuaCode
local function setDefs(varVal, selector, defVal)
  if not defVal then defVal = { } end
  varVal[selector] = varVal[selector] or defVal
  return varVal[selector]
end

litProgs.setDefs = setDefs
\stopLuaCode

\startTestCase[should set global defaults]
\startLuaTest
  local setDefs = thirddata.literateProgs.setDefs
  local test1Var = { }
  local test2Var = setDefs( test1Var, 'strSelector', 'aDefault')
  assert.isEqual(test1Var['strSelector'], 'aDefault')
  assert.isEqual(test2Var, 'aDefault')

  local test3Var = setDefs( test1Var, 'tableSelector')
  assert.isNotNil(test1Var['tableSelector'])
  assert.isTable(test1Var['tableSelector'])
  assert.length(test1Var['tableSelector'], 0)
  assert.isNotNil(test3Var)
  assert.isTable(test3Var)
  assert.length(test3Var, 0)
\stopLuaTest
\stopTestCase
\stopTestSuite

\startTestSuite[shouldExist]

\startLuaCode
local function shouldExist(varVal, selector, errorMessage)
  if not varVal[selector] then
    if not errorMessage then
      errorMessage = selector..' was not found but is required'
    end
    if type(errorMessage) == 'table' then
      errorMessage = tConcat(errorMessage)
    end
    error(errorMessage)
  end
  return varVal[selector]
end

litProgs.shouldExist = shouldExist

\stopLuaCode

\startTestCase[should succeed if selector exists]
\startLuaTest
  local shouldExist = thirddata.literateProgs.shouldExist
  local aVar = { }
  aVar.aSelector = 'something'
  assert.throwsNoError(
    shouldExist,
    'aSelector should exist',
    aVar,
    'aSelector'
  )
\stopLuaTest
\stopTestCase

\startTestCase[should require selector]
\startLuaTest
  local shouldExist = thirddata.literateProgs.shouldExist
  local aVar = { }
  assert.throwsError(
    shouldExist,
    'aSelector should NOT exist',
    aVar,
    'aSelector'
  )
\stopLuaTest
\stopTestCase
\stopTestSuite

\startLuaCode
local function markMkIVCodeOrigin()
  local codeType       = setDefs(code, 'MkIVCode')
  local codeStream     = setDefs(codeType, 'curCodeStream', 'default')
  codeStream           = setDefs(codeType, codeStream)
  return sFmt('%% from file: %s after line: %s',
    codeStream.fileName,
    toStr(
      mFloor(
        codeStream.startLine/code.lineModulus
      )*code.lineModulus
    )
  )
end

litProgs.markMkIVCodeOrigin = markMkIVCodeOrigin

local function markMpIVCodeOrigin()
  local codeType       = setDefs(code, 'MpIVCode')
  local codeStream     = setDefs(codeType, 'curCodeStream', 'default')
  codeStream           = setDefs(codeType, codeStream)
  return sFmt('%% from file: %s after line: %s',
    codeStream.fileName,
    toStr(
      mFloor(
        codeStream.startLine/code.lineModulus
      )*code.lineModulus
    )
  )
end

litProgs.markMpIVCodeOrigin = markMpIVCodeOrigin

local function markLuaCodeOrigin()
  local codeType       = setDefs(code, 'LuaCode')
  local codeStream     = setDefs(codeType, 'curCodeStream', 'default')
  codeStream           = setDefs(codeType, codeStream)
  return sFmt('-- from file: %s after line: %s',
    codeStream.fileName,
    toStr(
      mFloor(
        codeStream.startLine/code.lineModulus
      )*code.lineModulus
    )
  )
end

litProgs.markLuaCodeOrigin = markLuaCodeOrigin

local function markLuaTemplateOrigin()
  local codeType       = setDefs(code, 'LuaTemplate')
  local codeStream     = setDefs(codeType, 'curCodeStream', 'default')
  codeStream           = setDefs(codeType, codeStream)
  return sFmt('-- from file: %s after line: %s',
    codeStream.fileName,
    toStr(
      mFloor(
        codeStream.startLine/code.lineModulus
      )*code.lineModulus
    )
  )
end

litProgs.markLuaTemplateOrigin = markLuaTemplateOrigin

local function markCHeaderOrigin()
  local codeType       = setDefs(code, 'CHeader')
  local codeStream     = setDefs(codeType, 'curCodeStream', 'default')
  codeStream           = setDefs(codeType, codeStream)
  return sFmt('// from file: %s after line: %s',
    codeStream.fileName,
    toStr(
      mFloor(
        codeStream.startLine/code.lineModulus
      )*code.lineModulus
    )
  )
end

litProgs.markCHeaderOrigin = markCHeaderOrigin

local function markCCodeOrigin()
  local codeType       = setDefs(code, 'CCode')
  local codeStream     = setDefs(codeType, 'curCodeStream', 'default')
  codeStream           = setDefs(codeType, codeStream)
  return sFmt('// from file: %s after line: %s',
    codeStream.fileName,
    toStr(
      mFloor(
        codeStream.startLine/code.lineModulus
      )*code.lineModulus
    )
  )
end

litProgs.markCCodeOrigin = markCCodeOrigin
\stopLuaCode

\setLitProgsOriginMarker[MkIVCode][markMkIVCodeOrigin]
\setLitProgsOriginMarker[MpIVCode][markMpIVCodeOrigin]
\setLitProgsOriginMarker[LuaCode][markLuaCodeOrigin]
\setLitProgsOriginMarker[LuaTemplate][markLuaTemplateOrigin]

\section{Lua templates}

\startLuaTemplate
-- t-literate-progs templates

if not modules then modules = { } end
modules ['t-literate-progs-templates'] = {
    version   = 1.000,
    comment   = "Literate Programming in ConTeXt - templates",
    author    = "PerceptiSys Ltd (Stephen Gaito)",
    copyright = "PerceptiSys Ltd (Stephen Gaito)",
    license   = "MIT License"
}

thirddata               = thirddata               or {}
thirddata.literateProgs = thirddata.literateProgs or {}

local litProgs          = thirddata.literateProgs
litProgs.templates      = {}
local templates         = litProgs.templates
templates.mkiv          = {}
templates.lua           = {}
templates.templates     = {}
templates.lmsfile       = {}
templates.litProgsTable = {}

local table_insert = table.insert
local table_concat = table.concat

local addTemplate = litProgs.addTemplate
\stopLuaTemplate


\stopchapter
